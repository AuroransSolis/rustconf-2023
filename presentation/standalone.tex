%\documentclass[aspectratio = 169]{beamer}
\documentclass{beamer}

\usepackage{emoji}
\usepackage{fontawesome5}
\usepackage{fontspec}
\usepackage{framed}
\usepackage{hyperref}
\usepackage{iftex}
\usepackage{minted}
\usepackage{svg}
\usepackage{tabularray}
\usepackage{tikz}
\usepackage{ulem}
\usepackage{xcolor}

\ifpdftex
	\usepackage[utf8x]{inputenc}
\fi

%
% Document setup
%

% hyperref
\AtBeginDocument{
	\hypersetup{
		pdftitle = {Anything you can do, I can do worse with `macro\_rules!`},
		pdfauthor = {Aurorans Solis},
	}
}

% Fonts
\newfontfamily\jetbrainsmono{JetBrains Mono}[NFSSFamily = JetBrains Mono]
\setemojifont{TwemojiMozilla}

% Beamer theme
\usetheme[nofirafonts]{focus}
\setsansfont{Fira Sans}
\setmonofont{Fira Mono}

% Title
\title{Anything you can do, I can do worse with {\color{macrorulescolor}\texttt{macro\_rules!}}}
\author{Aurorans Solis}
\date{}
\titlegraphic{\includegraphics[scale = 0.15]{profile-pic.png}}

%
% Package configuration
%

% Minted
\setminted{
	bgcolor = lightergrey,
	fontfamily = JetBrains Mono,
	fontsize = \tiny,
	highlightcolor = blue9,
	linenos = true,
	tabsize = 4,
	numberblanklines = true,
	numbersep = 2pt,
}
\usemintedstyle{default}

% tikz
\usetikzlibrary{arrows, automata, positioning}

%
% Other
%

% Colours
\definecolor{discordcolor}{RGB}{114, 137, 218}
\definecolor{gitlabcolor}{RGB}{252, 109, 38}
\definecolor{lightergrey}{RGB}{230, 230, 230}
\definecolor{macrorulescolor}{RGB}{86, 156, 214}
\definecolor{lightgrey}{RGB}{212, 212, 212}
\definecolor{fragspeccolor}{RGB}{156, 220, 254}
\definecolor{ifcolor}{RGB}{197, 134, 192}
\definecolor{errwhite}{RGB}{255, 255, 255}
\definecolor{errredcolour}{RGB}{255, 0, 0}
\definecolor{errgreencolour}{RGB}{0, 255, 0}
\definecolor{errbluecolour}{RGB}{24, 135, 241}

% Extra commands
\newcommand{\fragspec}[1]{{\color{lightgrey}:}{\color{fragspeccolor}#1}}
\newcommand{\rustkeyword}[1]{{\color{ifcolor}\texttt{#1}}}
\newcommand{\rusttoken}[1]{{\color{lightgrey}\texttt{#1}}}
\newcommand{\eths}{\emoji{two-hearts}}
\newcommand{\maincolour}[1]{\colorbox{main}{#1}}
\newcommand{\errwhite}[1]{{\color{errwhitecolour}#1}}
\newcommand{\errred}[1]{{\color{errredcolour}#1}}
\newcommand{\errgreen}[1]{{\color{errgreencolour}#1}}
\newcommand{\errblue}[1]{{\color{errbluecolour}#1}}

\begin{document}
	\begin{frame}
		\maketitle
		\vspace{-1.9cm}
		\scriptsize{any/any}
		\vspace{1.75cm}
		\newline
		\small{\texttt{
			{\color{black}\faGithub} AuroransSolis\\
			{\color{gitlabcolor}\faGitlab} AuroransSolis\\
			{\color{discordcolor}\faDiscord} aurorans\_solis
		}}
		\vspace{4cm}
	\end{frame}

	\section{Quick Refresher}
	\begin{frame}{Easing Into It}
		A few quick things:
		\pause
		\begin{itemize}
			\item \texttt{@something} in a macro invocation isn't special syntax.

			\pause

			\item We use recursion a lot in macros because an individual rule itself cannot be
			recursive, ambiguous, or violate Rust's parsing ambiguity rules.
		\end{itemize}
	\end{frame}

	\begin{frame}{Hold on, that's a lot}
		\begin{enumerate}
			\item A rule cannot be recursive?
			\pause
			\begin{itemize}
				\item Rules only have regex-like repetition specifiers: \texttt{?}, \texttt{*}, and
				\texttt{+}. This doesn't give the tools for recursion.
			\end{itemize}

			\pause

			\item A rule cannot be ambiguous?
			\pause
			\begin{itemize}
				\item A rule must only be able to be applied in one way to a particular input. For
				example, \mintinline{rust}{($($foo:tt)? $bar:tt)} is ambiguous on the input
				\mintinline{rust}{baz}.

			\end{itemize}

			\pause

			\item A rule cannot violate Rust's parsing ambiguity rules?
			\pause
			\begin{itemize}
				\item You cannot have unpaired \texttt{()}, \texttt{[]}, or \texttt{\{\}}.
				\item You cannot have certain token types come before anything except certain allowed
				tokens (follow set ambiguity)
			\end{itemize}
		\end{enumerate}
	\end{frame}

	\section*{The Horrors}
	\begin{frame}{(:}
		\centering
		\includesvg[
			width = \linewidth,
			height = 0.8\textheight,
			inkscape = true,
			inkscapelatex = true,
			inkscapeformat = pdf,
			inkscapearea = page,
		]{time-1.svg}
	\end{frame}

	\begin{frame}{???}
		\centering
		\includesvg[
			width = \linewidth,
			height = 0.8\textheight,
			inkscape = true,
			inkscapelatex = true,
			inkscapeformat = pdf,
			inkscapearea = page,
		]{time-2.svg}
	\end{frame}

	\begin{frame}{!!!}
		\centering
		\includesvg[
			width = \linewidth,
			height = 0.8\textheight,
			inkscape = true,
			inkscapelatex = true,
			inkscapeformat = pdf,
			inkscapearea = page,
		]{time-3.svg}
	\end{frame}

	\begin{frame}{(:<}
		It's not snacktime anymore. It's crimetime.

		We're going to define traits using XML and declarative macros.
	\end{frame}

	\subsection{Start of the evil arc}
	\begin{frame}[fragile]{(:<}
		\begin{onlyenv}<1>
			We're going to take this trait:

			\begin{minted}[autogobble]{rust}
				pub trait Foo<const BAR: usize>: Baz {
					type Baq: Qux;
					const QUUX: Self::Baq;
					fn corge<Grault, Garply>(waldo: Grault) -> Garply;
				}
			\end{minted}
		\end{onlyenv}

		\pause

		\begin{onlyenv}<2->
			And turn it into this XML:
			\begin{center}
				\begin{minipage}{0.45\linewidth}
					\begin{minted}[autogobble]{xml}
						<trait>
							<name>Foo</name>
							<vis>pub</vis>
							<bounds>
								<const>
									<name>BAR</name>
									<type>usize</type>
								</const>
								<req>Baz</req>
							</bounds>
							<assoctype>
								<name>Baq</name>
								<bounds>
									<req>Qux</req>
								</bounds>
							</assoctype>
							<assocconst>
								<name>QUUX</name>
								<type>Self::Baq</type>
							</assocconst>
					\end{minted}
				\end{minipage}
				\hfill
				\begin{minipage}{0.45\linewidth}
					\begin{minted}[autogobble, firstnumber = 21]{xml}
							<assocfn>
								<name>corge</name>
								<bounds>
									<type>
										<name>Grault</name>
									</type>
									<type>
										<name>Garply</name>
									</type>
								</bounds>
								<args>
									<arg>
										<name>waldo</name>
										<type>Grault</type>
									</arg>
								</args>
								<ret>Garply</ret>
							</assocfn>
						</trait>
					\end{minted}
				\end{minipage}
			\end{center}

			\pause

			If you're asking, ``Why? Why would you do this to us?'' you're asking an excellent
			question. \pause I will not answer it.
		\end{onlyenv}
	\end{frame}

	\subsection{what}
	\begin{frame}{auro no please stop}
		See the better question is, ``Ooh, how does that work?'' \\

		\pause

		Let's look at a very simple example.
	\end{frame}

	\begin{frame}[fragile]{im begging you dont do this}
		\begin{onlyenv}<1>
			\begin{minted}[autogobble, linenos = false]{rust}
				trait_xml! {
					<trait>
						<name>Foo</name>
						<vis>pub</vis>
						<unsafe/>
					</trait>
				}
			\end{minted}
		\end{onlyenv}

		\pause

		\begin{onlyenv}<2>
			\begin{minted}[
				autogobble,
				escapeinside = ||,
				highlightlines = {1},
				linenos = false,
			]{rust}
				trait_xml_inner! {
					@parse {
						input: [
							|\errred{<trait>}|
								<name>Foo</name>
								<vis>pub</vis>
								<unsafe/>
							</trait>
						],
					}
				}
			\end{minted}
		\end{onlyenv}

		\pause

		\begin{onlyenv}<3>
			\begin{minted}[
				autogobble,
				escapeinside = ||,
				highlightlines = {2},
				linenos = false,
			]{rust}
				trait_xml_inner! {
					@parsetrait {
						input: [
								|\errred{<name>}|Foo</name>
								<vis>pub</vis>
								<unsafe/>
							</trait>
						],
						output: [],
					}
				}
			\end{minted}
		\end{onlyenv}

		\pause

		\begin{onlyenv}<4>
			\begin{minted}[
				autogobble,
				escapeinside = ||,
				highlightlines = {1, 10-16},
				linenos = false,
			]{rust}
				trait_xml_parse_name_ident! {
					@parse {
						input: [
								|\errred{Foo}|</name>
								<vis>pub</vis>
								<unsafe/>
							</trait>
						],
						name: ,
						callback: [
							name: trait_xml_inner,
							rule: [@namecallback],
							args: [
								output: [],
							],
						],
					}
				}
			\end{minted}
		\end{onlyenv}

		\pause

		\begin{onlyenv}<5>
			\begin{minted}[
				autogobble,
				escapeinside = ||,
				highlightlines = {2, 11-15},
				linenos = false,
			]{rust}
				trait_xml_parse_name_ident! {
					@parseend {
						input: [
								|\errred{</name>}|
								<vis>pub</vis>
								<unsafe/>
							</trait>
						],
						name: Foo,
						callback: [
							name: trait_xml_inner,
							rule: [@namecallback],
							args: [
								output: [],
							],
						],
					}
				}
			\end{minted}
		\end{onlyenv}

		\pause

		\begin{onlyenv}<6>
			\begin{minted}[autogobble, highlightlines = {1, 2, 9}, linenos = false]{rust}
				trait_xml_inner! {
					@namecallback {
						input: [
								<vis>pub</vis>
								<unsafe/>
							</trait>
						],
						output: [],
						name: Foo,
					}
				}
			\end{minted}
		\end{onlyenv}

		\pause

		\begin{onlyenv}<7>
			\begin{minted}[
				autogobble,
				escapeinside = ||,
				highlightlines = {2, 8},
				linenos = false,
			]{rust}
				trait_xml_inner! {
					@parsetrait {
						input: [
								|\errred{<vis>}|pub</vis>
								<unsafe/>
							</trait>
						],
						output: [[name Foo]],
					}
				}
			\end{minted}
		\end{onlyenv}

		\pause

		\begin{onlyenv}<8>
			\begin{minted}[
				autogobble,
				escapeinside = ||,
				highlightlines = {1, 8-14},
				linenos = false,
			]{rust}
				trait_xml_parse_vis! {
					@parse {
						input: [
								|\errred{pub</vis>}|
								<unsafe/>
							</trait>
						],
						callback: [
							name: trait_xml_inner,
							rule: [@viscallback],
							args: [
								output: [[name Foo]],
							],
						],
					}
				}
			\end{minted}
		\end{onlyenv}

		\pause

		\begin{onlyenv}<9>
			\begin{minted}[autogobble, linenos = false, highlightlines = {1, 2, 8}]{rust}
				trait_xml_inner! {
					@viscallback {
						input: [
								<unsafe/>
							</trait>
						],
						output: [[name Foo]],
						vis: [pub],
					}
				}
			\end{minted}
		\end{onlyenv}

		\pause

		\begin{onlyenv}<10>
			\begin{minted}[
				autogobble,
				escapeinside = ||,
				highlightlines = {2, 7},
				linenos = false,
			]{rust}
				trait_xml_inner! {
					@parsetrait {
						input: [
								|\errred{<unsafe/>}|
							</trait>
						],
						output: [[name Foo] [vis pub]],
					}
				}
			\end{minted}
		\end{onlyenv}

		\pause

		\begin{onlyenv}<11>
			\begin{minted}[
				autogobble,
				escapeinside = ||,
				highlightlines = {3},
				linenos = false,
			]{rust}
				trait_xml_inner! {
					@parsetrait {
						input: [|\errred{</trait>}|],
						output: [[name Foo] [vis pub] [unsafe]],
					}
				}
			\end{minted}
		\end{onlyenv}

		\pause

		\begin{onlyenv}<12>
			\begin{minted}[
				autogobble,
				escapeinside = ||,
				highlightlines = {2},
				linenos = false,
			]{rust}
				trait_xml_inner! {
					@expand {
						output: [|\errred{[name Foo]}| [vis pub] [unsafe]],
						vis: [],
						unsafe: ,
						name: ,
						gparams: [],
						tpbs: [],
						wc: [],
						assoc types: [],
						assoc consts: [],
						fns: [],
					}
				}
			\end{minted}
		\end{onlyenv}

		\pause

		\begin{onlyenv}<13>
			\begin{minted}[
				autogobble,
				escapeinside = ||,
				highlightlines = {6},
				linenos = false,
			]{rust}
				trait_xml_inner! {
					@expand {
						output: [|\errred{[vis pub]}| [unsafe]],
						vis: [],
						unsafe: ,
						name: Foo,
						gparams: [],
						tpbs: [],
						wc: [],
						assoc types: [],
						assoc consts: [],
						fns: [],
					}
				}
			\end{minted}
		\end{onlyenv}

		\pause

		\begin{onlyenv}<14>
			\begin{minted}[
				autogobble,
				escapeinside = ||,
				highlightlines = {4},
				linenos = false,
			]{rust}
				trait_xml_inner! {
					@expand {
						output: [|\errred{[unsafe]}|],
						vis: [pub],
						unsafe: ,
						name: Foo,
						gparams: [],
						tpbs: [],
						wc: [],
						assoc types: [],
						assoc consts: [],
						fns: [],
					}
				}
			\end{minted}
		\end{onlyenv}

		\pause

		\begin{onlyenv}<15>
			\begin{minted}[autogobble, linenos = false, highlightlines = {5}]{rust}
				trait_xml_inner! {
					@expand {
						output: [],
						vis: [pub],
						unsafe: unsafe,
						name: Foo,
						gparams: [],
						tpbs: [],
						wc: [],
						assoc types: [],
						assoc consts: [],
						fns: [],
					}
				}
			\end{minted}
		\end{onlyenv}

		\pause

		\begin{onlyenv}<16>
			\begin{minted}[autogobble, linenos = false]{rust}
				pub unsafe trait Foo {}
			\end{minted}
		\end{onlyenv}
	\end{frame}

	\subsection{why???}
	\begin{frame}{i will not apologise}
		There's a few points I'm trying to make here.

		\pause

		\begin{enumerate}
			\item Declarative macros are capable of a whole lot, actually

			\pause

			\item You don't always need a proc macro
			\begin{itemize}
				\item Simplify projects (don't need a separate crate)
				\item Reduce dependencies and sometimes compile times (no \texttt{syn} and
					\texttt{quote})
			\end{itemize}
		\end{enumerate}
	\end{frame}

	\section[Review of macro\_rules!]{Review of {\color{macrorulescolor}\texttt{macro\_rules!}}}
	\subsection{Why macros?}
	\begin{frame}{Why macros?}
		Turns out that metaprogramming is pretty cool actually \\

		\begin{itemize}
			\item Repeat lots of similar but not quite identical things
			\pause
			\item Define new grammars that get expanded to valid Rust
			\pause
			\item Confuse everyone (including yourself!)
		\end{itemize}
	\end{frame}

	\subsection[What is a macro\_rules!]{What is a {\color{macrorulescolor}\texttt{macro\_rules!}}?}
	\begin{frame}{What is a {\color{macrorulescolor}\texttt{macro\_rules!}}?}
		Declarative macros (the ones we care about for this presentation) are sort of like functions
		on the AST. \\

		\pause

		We try to match on certain AST patterns (rules) against the input. \\

		\pause

		Rules are tried in order from top to bottom. This is the arbiter between ambiguous rules.
		Both \mintinline{rust}{(foo bar)} and \mintinline{rust}{(foo $next:tt)} match
		\mintinline{rust}{foo bar}, but which one is chosen depends on their order in the macro def.
	\end{frame}

	\subsection{What are our AST node types, a.k.a. fragment specifiers?}
	\begin{frame}{Fragment specifier types}
		Rust has these fragment specifier types:
		\begin{center}
			\begin{tblr}{
					colspec = cccc,
					rowspec = cccc,
					row{1-4} = {bg = main},
				}
				\fragspec{item}
					& \fragspec{block}
					& \fragspec{stmt}
					& \fragspec{pat\_param} \\
				\fragspec{pat}
					& \fragspec{expr}
					& \fragspec{ty}
					& \fragspec{ident} \\
				\fragspec{path}
					& \eths\fragspec{tt}\eths
					& \fragspec{meta}
					& \fragspec{lifetime} \\
				\empty{}
					& \fragspec{vis}
					& \fragspec{literal}
					& \empty{}
			\end{tblr}
		\end{center}
		Each of these, except \maincolour{\fragspec{tt}}, are subject to regular Rust parsing rules.
		Plus regex-like repetitions with \texttt{\$()?}, \texttt{\$()*}, and \texttt{\$()+}. \\

		\pause

		There are also some limitations on what can come after certain fragment specifiers -- follow
		set ambiguity restrictions
		\begin{itemize}
			\item \maincolour{\fragspec{expr}} and \maincolour{\fragspec{stmt}} can only be followed
			by \texttt{=>}, \texttt{,}, or \texttt{;}

			\item \maincolour{\fragspec{pat\_param}} can only be followed by \texttt{=>},
			\texttt{,}, \texttt{=}, \texttt{|}, \rustkeyword{if}, or \rustkeyword{in}

			\item etc.
		\end{itemize}
	\end{frame}

	\subsection{Fragment Specifier Composition}
	\begin{frame}{Composition}
		Fragment specifiers can be composed into other fragment specifiers. For example, a
		\maincolour{\fragspec{ident}} and \maincolour{\fragspec{expr}} can be composed into a
		\maincolour{\fragspec{stmt}}. \\

		\pause

		The reverse is \textbf{NOT} true.
		\begin{center}
			\ttfamily
			\begin{tabular}{cc}
				\maincolour{\color{white}\fragspec{ident}, \fragspec{expr} => \fragspec{stmt}} =
				& \emoji{check-mark} \\
				\maincolour{\color{white}\fragspec{stmt} => \fragspec{ident}, \fragspec{expr}} =
				& \emoji{cross-mark}
			\end{tabular}
		\end{center}

		\pause

		However, \maincolour{\fragspec{tt}} tends to be the most flexible option for these sorts of
		operations.
	\end{frame}

	\begin{frame}[fragile]{Composition}
		\begin{minted}[autogobble, highlightlines = {4, 6}]{rust}
			macro_rules! me_reaping {
				($let:tt $lhs:tt $equal:tt $rhs:tt) => {
					// compose `:tt`s into a `:stmt`
					me_reaping!(@matchstmt $let $lhs $equal $rhs)
				};
				(@matchstmt $stmt:stmt) => {
					$stmt
				};
			}

			macro_rules! me_sowing {
				($stmt:stmt) => {
					// attempt to break a `:stmt` back into component `:tt`s
					me_reaping!($stmt);
				}
			}

			fn main() {
				me_reaping!(let haha = "yes!!!");
				println!("{haha}");
				me_sowing!(let well_this = "sucks ):");
				println!("{well_this}");
			}
		\end{minted}
	\end{frame}

	\begin{frame}{Composition}
		That gives the following error message:
		\begin{center}
			\colorbox{lightergrey}{
				\ttfamily
				\scriptsize
				\parbox{0.9\textwidth}{%
					\errred{error}: unexpected end of macro invocation\\
					\phantom{xx}\errblue{-->} src/main.rs:14:26\\
					\phantom{xxx}\errblue{|}\\
					\phantom{x}\errblue{1  |} macro\_rules! me\_reaping \{\\
					\phantom{xxx}\errblue{| ----------------------- when calling this macro}\\
					\errblue{...}\\
					\errblue{14 |}         me\_reaping!(\$stmt);\\
					\phantom{xxx}\errblue{|}\phantom{xxxxxxxxxxxxxxxxxx}\errred{\^{} missing tokens
						in macro arguments}\\
					\phantom{xxx}\errblue{|}\\
					\errgreen{note}: while trying to match meta-variable `\$lhs:tt`\\
					\phantom{xx}\errblue{-->} src/main.rs:2:14\\
					\phantom{xxx}\errblue{|}\\
					\phantom{x}\errblue{2  |}\phantom{xxxxx}(\$let:tt \$lhs:tt \$equal:tt \$rhs:tt)
						=> \{
					\phantom{xxx}\errblue{|}\phantom{xxxxxxxxxxxxxx}\errgreen{\^{}\^{}\^{}\^{}\^{}%
						\^{}\^{}}
				}
			}
		\end{center}

		\pause

		This is definitely all the magic stuff we will do with token composition (lies!)
	\end{frame}

	\begin{frame}{Main Restrictions}
		There's two main restrictions that I've come across that aren't super obvious at first:
		\pause
		\begin{enumerate}
			\item No significant whitespace
			\pause
			\item No matching tokens (cannot do \mintinline{rust}{$foo == $bar} or similar)

		\end{enumerate}
		\pause
		These are doable with proc macros, but those aren't the topic of today's talk.
	\end{frame}

	\section{Main Useful Patterns}
	\subsection{Overview}
	\begin{frame}{Overview}
		There are \sout{four}six Big Lads of the Macropalypse:
		\begin{itemize}
			\item Recursion
			\begin{itemize}
				\item of course macros can call themselves!
			\end{itemize}

			\pause

			\item Internal rules
			\begin{itemize}
				\item these are branches that generally shouldn't be called by users\
			\end{itemize}

			\pause

			\item Incremental TT munchers
			\begin{itemize}
				\item grabs chunks off the front end of the list of inputs
			\end{itemize}

			\pause

			\item Push-down accumulation
			\begin{itemize}
				\item holds tokens in a list for later expansion
			\end{itemize}

			\pause

			\item TT bundling
			\begin{itemize}
				\item boils down to grouping multiple tokens into a single list
			\end{itemize}

			\pause

			\item Callbacks
			\begin{itemize}
				\item workaround to let you pass the expansion of one macro as input to another*
			\end{itemize}
		\end{itemize}
	\end{frame}

	\begin{frame}{Overview}
		Let's look at each of these in turn, but first, disclaimers.

		\pause

		\begin{itemize}
			\item You're going to hear and see, ``You've seen this already in this presentation!''
				a lot.

			\pause

			\item<3->
				\only<3>{
					Yes, these patterns (in combination) will let you parse just about anything.
				}
				\only<4->{
					\sout{Yes, these patterns (in combination) will let you parse just about
					anything.} {\errred{STOP STOP STOP}}
				}

			\pause

			\item Recursion is \textbf{THE} building block for macros using the aforementioned
				patterns, so for big inputs you may end up needing
				{\small\texttt{\#![recursion\_limit = "a very big number"]}}.
				\pause
				And a long time to compile.
				\pause
				And a lot of memory.
		\end{itemize}
	\end{frame}

	\begin{frame}{Overview}
		Let's look at each of these in turn, but first, disclaimers.

		\begin{itemize}
			\item Declarative macros can be (and often are) very difficult to debug.

			\pause

			\item Maintenance of big macros is\dots{} oh boy.

			\pause

			\item All that said, these patterns can be leveraged to simplify some things quite a
				lot.

			\pause

			\item This talk is mostly going to be showing how some pretty cursed stuff works, but
				at the end of the talk I'll give you a few ways to help ``debug'' macros and make
				writing them more manageable.
		\end{itemize}
	\end{frame}

	\subsection{Recursion}
	\begin{frame}[fragile]{Recursion my beloved}
		You've seen this already in this presentation! \\

		This is the tool that every other pattern mentioned uses to work.

		\begin{minted}[autogobble, highlightlines = {3}, linenos = false]{rust}
			macro_rules! some_macro {
				/* pattern */ => {
					some_macro!(/* ... */)
				};
			}
		\end{minted}
		In function-like syntax, this is roughly equivalent to:
		\begin{minted}[autogobble, linenos = false]{rust}
			fn some_macro(input: Ast) -> Ast {
				match input {
					/* pattern */ => some_macro(/* ... */)
				}
			}
		\end{minted}
	\end{frame}

	\subsection{Internal Rules}
	\begin{frame}[fragile]{Internal rules my beloved}
		You've seen this already in this presentation! Two slides prior, even! \\

		As said previously, you generally don't want users calling these rules. Usually they're used
		as a helper to grab new kinds of tokens or to specify a certain mode of parsing.

		\begin{minted}[autogobble, highlightlines = {3}, linenos = false]{rust}
			macro_rules! some_macro {
				/* other rules */
				(@internalrule /* finish pattern */) => {
					/* expansion */
				};
			}
		\end{minted}

		Full examples:
		\begin{itemize}
			\item \path{fragments-demo/src/main.rs:[4,6]}
			\item \path{all-unique/src/main.rs:[5,9,12,15]}
			\item \path{reverse-tokens/src/main.rs:[2,7]}
		\end{itemize}
	\end{frame}

	\begin{frame}{Internal rules my beloved}
		Not all internal rules start with \texttt{@something}! Internal rules are just any rules
		that users are not expected to call but are used at some intermediate stage in macro
		expansion. \\

		Useful in a couple ways
		\begin{itemize}
			\item Help avoid polluting crate namespace
			\begin{itemize}
				\item Each internal rule \emph{could} be its own macro, but those would also have to
					be marked with \texttt{\#[macro\_export]}
			\end{itemize}

			\item Can be used to set ``modes''
			\begin{itemize}
				\item Useful for parsing context-sensitive things
			\end{itemize}
		\end{itemize}
	\end{frame}

	\subsection{Incremental TT munchers}
	\begin{frame}[fragile]{Incremental TT munchers my beloved}
		You haven't seen this one already in this presentation! Surprise! \\

		With these you typically look to grab an expected pattern including some inputs.
		\begin{minted}[autogobble, highlightlines = {2, 3, 4, 6}, linenos = false]{rust}
			macro_rules! some_macro {
				() => { /* expansion */ };
				($first:tt $($rest:tt)*) => {
					do_something_with!($first);
					/* expansion */
					some_macro!($($rest)*);
				};
			}
		\end{minted}
		As a function, this kinda looks like:
		\begin{minted}[autogobble, linenos = false]{rust}
			fn some_macro(input: Ast) -> Ast {
				if input.is_empty() {
					/* expansion */
				} else {
					do_something_with(input[0]);
					/* expansion */
					some_macro(input[1..])
				}
			}
		\end{minted}
	\end{frame}

	\begin{frame}{Incremental TT munchers my beloved}
		Full examples of incremental TT munchers:
		\begin{itemize}
			\item \path{all-unique/src/main.rs:[1-18]}
			\item \path{muncher-demo/src/main.rs:[1-9]}
		\end{itemize}
	\end{frame}

	\subsection{Push-down Accumulation}
	\begin{frame}[fragile]{Push-down accumulation my beloved}
		This one hasn't been shown in this presentation yet! \\

		Macros have to expand to a complete syntax element. Push-down accumulation is how you can
		allow for incomplete intermediate expansions that eventually form a complete expansion.
		For example:
		\begin{minted}[autogobble, highlightlines = {3}, linenos = false]{rust}
			macro_rules! some_macro {
				/* other rules */
				([$name:ident $($t:ty)+]) => {
					pub struct $name($($t),+);
				}
			}
		\end{minted}

		Full example: \path{reverse-tokens/src/main.rs:[2,4,7,12]}
	\end{frame}

	\subsection{TT Bundling}
	\begin{frame}[fragile]{TT bundling my beloved}
		This one hasn't shown up yet either! \\

		TT bundling is a sort of special case for composition, except this time we \textit{can}
		actually reverse it! \\

		Multiple tokens $\Rightarrow$ \texttt{[]}-list (\fragspec{tt})

		\inputminted[
			autogobble,
			highlightlines = {4, 7, 13},
			lastline = 16,
			linenos = false,
		]{rust}{../tt-bundling/src/main.rs}
	\end{frame}

	\subsection{Callbacks}
	\begin{frame}[fragile]{Callbacks my beloved}
		Very very generally, a callback looks something like this:

		\begin{minted}[autogobble, highlightlines = {2, 3}, linenos = false]{rust}
			macro_rules! callback {
				($callback:ident( $($args:tt)* )) => {
					$callback!( $($args)* )
				};
			}
		\end{minted}

		\pause

		Provided you have a consistent framework for calling your macros, these let you:
		\begin{itemize}
			\item Reuse one macro in multiple places
			\item Form something like a call stack
			\item \emph{\textbf{Recursively}} pass the expansion of one macro as input to another
		\end{itemize}

		These are great for being able to reuse one macro in multiple places.\pause{} As long as
		you have a consistent framework for calling your macros.
	\end{frame}

	\section{Debugging}
	\begin{frame}{That's the neat part}
		\includegraphics[width = \linewidth]{neat-part.png}
	\end{frame}

	\begin{frame}[fragile]{Kinda, at least}
		If you're defining items outside of functions, how do you debug things? \\

		\pause

		What do when no \mintinline{rust}{println!}? \\

		\pause

		Introducing your new best friend:
		\begin{minted}[autogobble, linenos = false, highlightlines = {1, 3}]{rust}
			const _: &str = stringify!($token);
			// also
			const _: &str = concat!($(stringify($tokens)),*);
		\end{minted}
	\end{frame}

	\begin{frame}[fragile]{Pretend println}
		You can see some residue from me debugging things and creating examples for documentation
		in \path{trait_xml_macro.rs}:
		\inputminted[
			firstline = 361,
			lastline = 383,
			highlightlines = {368},
		]{rust}{../trait-xml/src/trait_xml_macro.rs}
	\end{frame}

	\begin{frame}[fragile]{No rules expected the token\dots}
		``But Auro! My macro just plain doesn't work! I can't use \mintinline{rust}{const _}.'' \\

		Yes you can.

		\begin{minted}[autogobble, linenos = false, highlightlines = {4-6}]{rust}
			macro_rules! this_fails_somehow {
				// a bunch of rules up here...
				// ...and then at the end:
				($($all:tt)*) => {
					const _: &str = concat!($(stringify!($all)),*);
				};
			}
		\end{minted}

		This shows you the input as a string so you can try and figure out what's going wrong. \\

		It's also only matched if no other branch matches. But\dots
		\pause
		You still have to figure out why it's not matching.
	\end{frame}

	\begin{frame}[fragile]{No rules expected the token\dots}
		Another thing you can do to keep \texttt{\errred{error}: no rules expected the token} from
		coming up is by just\dots{} making rules that expect the token/those tokens. \\

		\pause

		My XML parsing macros do this a bunch - if you give them an invalid input, a lot of the
		time it'll expand to a \texttt{compile\_error!}, e.g. in \texttt{lifetime.rs}:
		\inputminted[
			firstline = 342,
			lastline = 354,
			highlightlines = {354},
		]{rust}{../trait-xml/src/lifetime.rs}
	\end{frame}

	\begin{frame}{Other things}
		Some other suggestions to help you on your way (that sound a lot like general programming
		advice):
		\begin{itemize}
			\item Use descriptive names for rules, helper tokens, and fragments
			\item Don't worry about excessive complexity in a single rule -- just try to keep the
				depth of recursion down if you can
			\item Don't be afraid to make another macro, especially if it can be reused in other
				places
			\item If you need more steps to finish parsing, add them. Nobody needs to see all your
				internal rules but you (:
		\end{itemize}
	\end{frame}

	\section{That's all!}
	\begin{frame}{Materials}
		The materials for this talk are available on GitHub and GitLab at
		\texttt{AuroransSolis/rustconf-2023}.
	\end{frame}
\end{document}
